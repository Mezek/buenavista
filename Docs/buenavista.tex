%% -*- TeX -*-
%% File: `buenavista.tex'
%% Coding: utf-8
%% Format: pdfLaTeX
%% Description: Notes to Unitary and Analytic model
%% Date: Tue 06 Nov 2012 01:47:21 PM CET 
%% (C) 2012 Erik Bartoš
%%%%%%%%%%%%%%%%%%%%%%%%%%%%%%%%%%%%%%%%%%%%%%%%%%%%%%%%%%%%%%%%%%%%%%%%%%%%%%%%

\documentclass[a4paper,draft]{article}
% Adjust final/draft for final version!

\usepackage{helvet}
%\usepackage[slovak]{babel}
\usepackage[utf8x]{inputenc}
\usepackage[T1]{fontenc}
%\usepackage{lmodern}
%\usepackage{indentfirst}

\usepackage{amsmath,amssymb}
\usepackage{hepnames}
\usepackage{booktabs}
\usepackage{multirow}
\usepackage{array}
\usepackage{adjustbox,ragged2e}
\usepackage{siunitx}

\newcommand*\doctitle{Notes to Unitary and Analytic model}
\newcommand*\docauthor{Erik Bartoš}
\newcommand*\docdate{\today}

\usepackage[pdftex]{hyperref}
\hypersetup{
      breaklinks    = true,
      baseurl       = http://,
      pdfborder     = 0 0 0,
      pdfpagemode   = UseNone, %UseNone, UseOutline
%      pdfstartview  = FitH,
      pdfstartpage  = 1,
      pdfcreator    = Pdf\LaTeX{} with hyperref package,
      pdfproducer   = Pdf\LaTeX{},
      bookmarksopen = true,
      pdfauthor     = {\docauthor},
      pdftitle      = {\doctitle},  
      pdfsubject    = {Article},
      pdfkeywords   = {scientific,paper}
}

\usepackage{graphicx}
\usepackage{epstopdf}
%\DeclareGraphicsExtensions{.eps}
%\epstopdfsetup{suffix=}

\usepackage[colorinlistoftodos, textwidth=4cm, shadow]{todonotes}
% option: disable - disable todonotes package
% option: obeyDraft - first check draft option
%%\usepackage[displaymath, tightpage]{preview}
\usepackage{setspace,fancyhdr}


%\textwidth=0.6\paperwidth

% Declarations math
\newcommand*{\m}[1]{\mathrm {#1}}
\newcommand*{\dd}{\ensuremath{\mathrm{d}}}
\newcommand*{\ee}{\ensuremath{\mathrm{e}}}
\newcommand*{\ii}{\ensuremath{\mathrm{i}}}
\newcommand*{\nn}{\nonumber}
\newcommand*{\G}[3][G]{\ensuremath{#1_{\m{#2}}^{\m{#3}}}}
\newcommand*{\F}[2]{\ensuremath{\G{#1}{#2}}}

\DeclareRobustCommand{\APN}{\HepAntiParticle{N}{}{}\xspace}
\DeclareRobustCommand{\APgL}{\HepAntiParticle{\Lambda}{}{}\xspace}
\DeclareRobustCommand{\PgXz}{\HepParticle{\Xi}{}{0}\xspace}
\DeclareRobustCommand{\APgXz}{\HepAntiParticle{\Xi}{}{0}\xspace}
\DeclareRobustCommand{\Pgfb}{\HepParticleResonance{\Pgf}{2175}{}{}\xspace}
\DeclareRobustCommand{\Pgrdva}{\HepParticleResonance{\Pgr}{1570}{}{}\xspace}
\DeclareRobustCommand{\Pgrtri}{\HepParticleResonance{\Pgr}{1720}{}{}\xspace}
\DeclareRobustCommand{\Pgos}{\HepParticle{\Pgo'}{}{}\xspace}
\DeclareRobustCommand{\Pgoss}{\HepParticle{\Pgo''}{}{}\xspace}
\DeclareRobustCommand{\Pgfs}{\HepParticle{\Pgf'}{}{}\xspace}
\DeclareRobustCommand{\Pgfss}{\HepParticle{\Pgf''}{}{}\xspace}
\DeclareRobustCommand{\Pgrs}{\HepParticle{\Pgr'}{}{}\xspace}
\DeclareRobustCommand{\Pgrss}{\HepParticle{\Pgr''}{}{}\xspace}

\newcommand*{\bra}{\mathinner\langle}
\newcommand*{\ket}{\mathinner\rangle}
\newcommand*{\ack}{\mathinner\vert}
\newcommand*{\radE}[1][2]{\bra r_{\m{E}}^{#1}\ket}
\newcommand*{\radM}[1][2]{\bra r_{\m{M}}^{#1}\ket}

% Declarations todonotes
\newcommand{\todoGreen}[1]{\todo[color=green!40]{#1}}

% Declarations units
\DeclareSIUnit \GeVc {\giga\electronvolt\per\textit{c}}
\DeclareSIUnit \GeVcS {(\GeVc)\squared}
\DeclareSIUnit \fm {\femto\meter }

\def\tol#1#2#3{\hbox{\rule{0pt}{12pt}${#1}_{-{#2}}^{+{#3}}$}}

\newcounter{mycomment}
\newcommand*{\todoA}[2][]{% 
	% initials of the author (optional) + note in the margin
	\refstepcounter{mycomment}%
	{%
		\setstretch{0.7}% spacing
		\todo[color={red!100!green!33},size=\small]{%
		%\textbf{Comment [\uppercase{#1}\themycomment]:}~#2}%
		\textbf{[\uppercase{#1}\themycomment]:}~#2}%
	}}
\newcommand*{\todoEB}[2][]{\todoA[EB#1]{#2}}

% Declarations fancyhdr
\pagestyle{fancy}
\fancyhf{} % clear all header and footer fields
\renewcommand{\headrulewidth}{0pt}
%\renewcommand{\footrulewidth}{0pt}
\fancyhead[L]{\footnotesize\parbox{0.7\textwidth}{\doctitle}}
\fancyhead[R]{\footnotesize\docauthor}
%\fancyfoot[C]{\docauthor}
\fancyfoot[C]{\footnotesize Page \thepage\ of 2}
\fancypagestyle{plain}{\fancyhf{}} % first page

%%%%%%%%%%%%%%%%%%%%%%%%%%%%%%%%%%%%%%%%%%%%%%%%%%%%%%%%%%%%%%%%%%%%%%%%%%%%%%%%
\begin{document}

\title{\doctitle}
\author{\docauthor}

\date{\docdate}
\maketitle


\begin{abstract}
We have analyzed the electromagnetic form factors of $1/2^{+}$ baryon octet.
\end{abstract}

\listoftodos

%%%%%%%%%%%%%%%%%%%%%%%%%%%%%%%%%%%%%%%%%%%%%%%%%%%%%%%%%%%%%%%%%%%%%%%%%%%%%%%%
\section{Intro}
%%%%%%%%%%%%%%%%%%%%%%%%%%%%%%%%%%%%%%%%%%%%%%%%%%%%%%%%%%%%%%%%%%%%%%%%%%%%%%%%

The notes done during the calculations and in the time of preparing literature.
\todoGreen{Example: A green note.}
\todoEB{An author comment.}

%%%%%%%%%%%%%%%%%%%%%%%%%%%%%%%%%%%%%%%%%%%%%%%%%%%%%%%%%%%%%%%%%%%%%%%%%%%%%%%%
\section{Cross sections}
%%%%%%%%%%%%%%%%%%%%%%%%%%%%%%%%%%%%%%%%%%%%%%%%%%%%%%%%%%%%%%%%%%%%%%%%%%%%%%%%

The standard dipole\footnote{sd --- standard dipole}

\begin{equation}
\G{sd}{}(Q^2) = \bigg(1+\frac{Q^2}{\SI{0.71}{\GeVcS}}\bigg)^2
\end{equation}

%%%%%%%%%%%%%%%%%%%%%%%%%%%%%%%%%%%%%%%%%%%%%%%%%%%%%%%%%%%%%%%%%%%%%%%%%%%%%%%%
\section{Proton electromagnetic radii}
%%%%%%%%%%%%%%%%%%%%%%%%%%%%%%%%%%%%%%%%%%%%%%%%%%%%%%%%%%%%%%%%%%%%%%%%%%%%%%%%

\begin{table}[!ht]\centering
\begin{tabular}{
	l
	l
	l
	l
	S[table-number-alignment = left]
}
\toprule
{Paper} & {$\radE[]$} & {$\radE$} & {$\radM[]$} & {$\radM$}\\
\midrule
{\cite{Mohr:2008fa}} & 0.8768(69) & 0.7688 & &\\
{\cite{Pohl:2010zza}} & 0.84184(67) & 0.70869 & &\\
{\cite{Lorenz:2012tm}} & \tol{0.84}{0.01}{0.01} & 0.7056 & \tol{0.86}{0.02}{0.01} &\\
{\cite{Antognini:1900ns}} & 0.84087(39) & 0.70706 & 0.87(6) & \\
\midrule
\bottomrule
\end{tabular}
\caption{The values of proton electromagnetic radii are in \si{\fm}.}
\label{tab:radii}
\end{table}

%%%%%%%%%%%%%%%%%%%%%%%%%%%%%%%%%%%%%%%%%%%%%%%%%%%%%%%%%%%%%%%%%%%%%%%%%%%%%%%%
\section{Hyperons}
%%%%%%%%%%%%%%%%%%%%%%%%%%%%%%%%%%%%%%%%%%%%%%%%%%%%%%%%%%%%%%%%%%%%%%%%%%%%%%%%

Baryon form factors
\begin{align}
\G{E\Pp(\Pn)}{}(t)&=\G{E\PN}{s}(t)\pm\G{E\PN}{v}(t)\nn\\
\G{M\Pp(\Pn)}{}(t)&=\G{M\PN}{s}(t)\pm\G{M\PN}{v}(t)\nn\\
\G{E\PgL}{}(t)&=\G{E\PgL}{s}(t)\nn\\
\G{M\PgL}{}(t)&=\G{M\PgL}{s}(t)\nn\\
\G{E\PgSp(\PgSm)}{}(t)&=\G{E\PgS}{s}(t)\pm\G{E\PgS}{v}(t)\nn\\
\G{M\PgSp(\PgSm)}{}(t)&=\G{M\PgS}{s}(t)\pm\G{M\PgS}{v}(t)\nn\\
\G{E\PgSz}{}(t)&=\G{E\PgSz}{s}(t)\nn\\
\G{M\PgSz}{}(t)&=\G{M\PgSz}{s}(t)\nn\\
\G{E\PgXz(\PgXm)}{}(t)&=\G{E\PgX}{s}(t)\pm\G{E\PgX}{v}(t)\nn\\
\G{M\PgXz(\PgXm)}{}(t)&=\G{M\PgX}{s}(t)\pm\G{M\PgX}{v}(t)
\end{align}

\begin{table}[!ht]\centering
\resizebox{\textwidth}{!}{%
\begin{tabular}{ccccccc}
\toprule
Baryon & $\G{E}{}$ & $\G{M}{}$ & $\G{E}{s}$ & $\G{E}{v}$ & $\G{M}{s}$ & $\G{M}{v}$\\
\midrule
\Pp   &  1 & $\mu_{\Pp}$    & \multicolumn{1}{c}{\multirow{2}{*}{$\tfrac{1}{2}$}} & \multicolumn{1}{c}{\multirow{2}{*}{$\tfrac{1}{2}$}} & \multicolumn{1}{c}{\multirow{2}{*}{$\tfrac{1}{2}(\mu_{\Pp}+\mu_{\Pn})$}} & \multicolumn{1}{c}{\multirow{2}{*}{$\tfrac{1}{2}(\mu_{\Pp}-\mu_{\Pn})$}} \\
\Pn   &  0 & $\mu_{\Pn}$      &  &  &  &  \\
\PgL  &  0 & $\mu_{\PgL}$     & 0 & --- & $\mu_{\PgL}$ & --- \\ 
\PgSp &  1 & $\mu_{\PgSp}$  & \multicolumn{1}{c}{\multirow{2}{*}{0}} & \multicolumn{1}{c}{\multirow{2}{*}{1}} & \multicolumn{1}{c}{\multirow{2}{*}{$\tfrac{1}{2}(\mu_{\PgSp}+\mu_{\PgSm})$}} & \multicolumn{1}{c}{\multirow{2}{*}{$\tfrac{1}{2}(\mu_{\PgSp}-\mu_{\PgSm})$}} \\
\PgSm & -1 & $\mu_{\PgSm}$ &  &  &  &  \\
\PgSz &  0 & $\tfrac{1}{2}(\mu_{\PgSp}+\mu_{\PgSm})$ & 0 & --- & $\tfrac{1}{2}(\mu_{\PgSp}+\mu_{\PgSm})$ & --- \\
\PgXz &  0 & $\mu_{\PgXz}$    & \multicolumn{1}{c}{\multirow{2}{*}{$-\tfrac{1}{2}$}} & \multicolumn{1}{c}{\multirow{2}{*}{$\tfrac{1}{2}$}} & \multicolumn{1}{c}{\multirow{2}{*}{$\tfrac{1}{2}(\mu_{\PgXz}+\mu_{\PgXm})$}} & \multicolumn{1}{c}{\multirow{2}{*}{$\tfrac{1}{2}(\mu_{\PgXz}-\mu_{\PgXm})$}} \\
\PgXm & -1 & $\mu_{\PgXm}$ &  &  &  &  \\
\bottomrule
\end{tabular}
}
\caption{The baryon norms, i.~e, the values of form factors in $t=0$.}
\label{tab:norms}
\end{table}

and their decomposition to Dirac and Pauli parts
\begin{align}
\G{E\Pp(\Pn)}{}(t)&=[\F{1\PN}{s}(t)\pm\F{1\PN}{v}(t)]+\frac{t}{4m_{\PN}^2}[\F{2\PN}{s}(t)\pm\F{2\PN}{v}(t)]\nn\\
\G{M\Pp(\Pn)}{}(t)&=[\F{1\PN}{s}(t)\pm\F{1\PN}{v}(t)]+[\F{2\PN}{s}(t)\pm\F{2\PN}{v}(t)]\nn\displaybreak\\
\G{E\PgL}{}(t)&=\F{1\PgL}{s}(t)+\frac{t}{4m_{\PgL}^2}\F{2\PgL}{s}(t)\nn\\
\G{M\PgL}{}(t)&=\F{1\PgL}{s}(t)+\F{2\PgL}{s}(t)\nn\\
\G{E\PgSp(\PgSm)}{}(t)&=[\F{1\PgS}{s}(t)\pm\F{1\PgS}{v}(t)]+\frac{t}{4m_{\PgS}^2}[\F{2\PgS}{s}(t)\pm\F{2\PgS}{v}(t)]\nn\\
\G{M\PgSp(\PgSm)}{}(t)&=[\F{1\PgS}{s}(t)\pm\F{1\PgS}{v}(t)]+[\F{2\PgS}{s}(t)\pm\F{2\PgS}{v}(t)]\nn\\
\G{E\PgSz}{}(t)&=\F{1\PgSz}{s}(t)+\frac{t}{4m_{\PgSz}^2}\F{2\PgSz}{s}(t)\nn\\
\G{M\PgSz}{}(t)&=\F{1\PgSz}{s}(t)+\F{2\PgSz}{s}(t)\nn\\
\G{E\PgXz(\PgXm)}{}(t)&=[\F{1\PgX}{s}(t)\pm\F{1\PgX}{v}(t)]+\frac{t}{4m_{\PgX}^2}[\F{2\PgX}{s}(t)\pm\F{2\PgX}{v}(t)]\nn\\
\G{M\PgXz(\PgXm)}{}(t)&=[\F{1\PgX}{s}(t)\pm\F{1\PgX}{v}(t)]+[\F{2\PgX}{s}(t)\pm\F{2\PgX}{v}(t)].
\end{align}

\begin{table}[!ht]\centering
{\scriptsize
\begin{tabular}{ccccccc}
\toprule
Baryon & $\G{E}{}$ & $\G{M}{}$ & $\F{1}{s}$ & $\F{1}{v}$ & $\F{2}{s}$ & $\F{2}{v}$\\
\midrule
\Pp   &  1 & $\mu_{\Pp}$    & \multicolumn{1}{c}{\multirow{2}{*}{$\tfrac{1}{2}$}} & \multicolumn{1}{c}{\multirow{2}{*}{$\tfrac{1}{2}$}} & \multicolumn{1}{c}{\multirow{2}{*}{$\tfrac{1}{2}(\mu_{\Pp}+\mu_{\Pn}-1)$}} & \multicolumn{1}{c}{\multirow{2}{*}{$\tfrac{1}{2}(\mu_{\Pp}-\mu_{\Pn}-1)$}} \\
\Pn   &  0 & $\mu_{\Pn}$      &  &  &  &  \\
\PgL  &  0 & $\mu_{\PgL}$     & 0 & --- & $\mu_{\PgL}$ & --- \\ 
\PgSp &  1 & $\mu_{\PgSp}$  & \multicolumn{1}{c}{\multirow{2}{*}{0}} & \multicolumn{1}{c}{\multirow{2}{*}{1}} & \multicolumn{1}{c}{\multirow{2}{*}{$\tfrac{1}{2}(\mu_{\PgSp}+\mu_{\PgSm})$}} & \multicolumn{1}{c}{\multirow{2}{*}{$\tfrac{1}{2}(\mu_{\PgSp}-\mu_{\PgSm}-2)$}} \\
\PgSm & -1 & $\mu_{\PgSm}$ &  &  &  &  \\
\PgSz &  0 & $\tfrac{1}{2}(\mu_{\PgSp}+\mu_{\PgSm})$ & 0 & --- & $\tfrac{1}{2}(\mu_{\PgSp}+\mu_{\PgSm})$ & --- \\
\PgXz &  0 & $\mu_{\PgXz}$    & \multicolumn{1}{c}{\multirow{2}{*}{$-\tfrac{1}{2}$}} & \multicolumn{1}{c}{\multirow{2}{*}{$\tfrac{1}{2}$}} & \multicolumn{1}{c}{\multirow{2}{*}{$\tfrac{1}{2}(\mu_{\PgXz}+\mu_{\PgXm}+1)$}} & \multicolumn{1}{c}{\multirow{2}{*}{$\tfrac{1}{2}(\mu_{\PgXz}-\mu_{\PgXm}-1)$}} \\
\PgXm & -1 & $\mu_{\PgXm}$ &  &  &  &  \\
\bottomrule
\end{tabular}}
\caption{The baryon norms, i.~e., the values of form factors in $t=0$.}
\label{tab:dnorms}
\end{table}

The lepton width of a vector mesons is defined\todo{$V$ in $\Gamma \equiv$ vector mesons?!}

\begin{equation}
\Gamma(V\to \Pep\Pem)= (\alpha^2 m_i/3)(f_i^2/4\pi),\quad i = S,V
\end{equation}
where the masses and widths are taken from
\begin{itemize}
\item all basic masses and widths -- \cite{Nakamura:2010zzi}

\item $\Pgos$, $\Pgoss$ -- Model A \cite{Donnachie:1988ws}

\item $\Pgrs$ width -- estimated from $\Gamma_{\Pgos\to\Pep\Pem}=\tfrac{1}{9}\Gamma_{\Pgrs\to\Pep\Pem}$ \cite{Donnachie:1988ws}

\item $\Pgrss$ width -- estimated from $\Gamma_{\Pgoss\to\Pep\Pem}=\tfrac{1}{9}\Gamma_{\Pgrss\to\Pep\Pem}$ \cite{Donnachie:1988ws}

\item $\Pgf$ sign is $-1$.\todo{Find in literature.}
\end{itemize}

The mean square radius of the charge distribution is given by
\begin{equation}
<r^2>=4\pi\int\limits_0^\infty r^2\rho(r) \dd r=\frac{6}{F(0)}\frac{\dd F(t)}{\dd t}\Big|_{t\to 0}
\end{equation}
for the case of neutral baryons, the radii is defined as
\begin{equation}
<r^2>=6\frac{\dd F(t)}{\dd t}\Big|_{t\to 0}.
\end{equation}


%%%%%%%%%%%%%%%%%%%%%%%%%%%%%%%%%%%%%%%%%%%%%%%%%%%%%%%%%%%%%%%%%%%%%%%%%%%%%%%%
\section{Calculations}
%%%%%%%%%%%%%%%%%%%%%%%%%%%%%%%%%%%%%%%%%%%%%%%%%%%%%%%%%%%%%%%%%%%%%%%%%%%%%%%%

\begin{table}[!ht]\centering
\begin{tabular}{cl}
\toprule
Coupling constant & Value \\
\midrule
$f_{\Pgo}$ & 17.0576 \\
$f_{\Pgf}$ & 13.4448 \\
$f_{\Pgr}$ & 4.95691 \\
\midrule
\multicolumn{2}{c}{0.788198 : 0.290599 : 0.368687} \\
\midrule
$f_{\Pgos}$ & 47.5897 \\
$f_{\Pgfs}$ &  \\
$f_{\Pgrs}$ & 13.6455 \\
\midrule
\multicolumn{2}{c}{--- : 0.2867 : ---} \\
\midrule
$f_{\Pgoss}$ & 48.3651 \\
$f_{\Pgfss}$ &  \\
$f_{\Pgrss}$ & 22.52751 \\
\midrule
\multicolumn{2}{c}{--- : 0.4658 : ---} \\
\bottomrule
\end{tabular}
\caption{The values of universal coupling constants.\todo[inline]{Sign of $f_\phi$ is $\pm$?}}
\label{tab:fmesons}
\end{table}

%%%%%%%%%%%%%%%%%%%%%%%%%%%%%%%%%%%%%%%%%%%%%%%%%%%%%%%%%%%%%%%%%%%%%%%%%%%%%%%%
\section{Results}
%%%%%%%%%%%%%%%%%%%%%%%%%%%%%%%%%%%%%%%%%%%%%%%%%%%%%%%%%%%%%%%%%%%%%%%%%%%%%%%%

\begin{table}[!ht]\centering
\begin{tabular}{cccc}
\toprule
Baryon & $\chi$PT %\cite{Kubis:2000aa}
 & Our & Exp\\
\midrule
\Pp & 0. & 0. & \\
\Pn & 0. & 0. & \\
\PgL & 0. & 0. & \\
\PgSp & 0. & 0. & \\
\PgSz & 0. & 0. & \\
\PgSm & 0. & 0. & $0.61\pm 0.12\pm 0.09$\\
\PgXz & 0. & 0. & \\
\PgXm & 0. & 0. & \\
\bottomrule
\end{tabular}
\caption{The electric radii of baryons.}
\label{tab:rade}
\end{table}

\begin{figure}[t]
\missingfigure{AaAaA}
%\includegraphics[width=\columnwidth]{fig01}
\caption{}
\label{fig:label0}
\end{figure}


\begin{footnotesize}
\bibliography{hyperons}\bibliographystyle{apalike}
\end{footnotesize}

\end{document}
%% End of file
